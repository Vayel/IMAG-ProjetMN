\documentclass[12pt]{article}

\usepackage{amsfonts, amsmath, amssymb, amstext, latexsym}
\usepackage{graphicx, epsfig}
\usepackage[latin1]{inputenc}
\usepackage[french]{babel}
\usepackage{exscale}
\usepackage{amsbsy}
\usepackage{amsopn}
\usepackage{fancyhdr}

\newcommand{\noi}{\noindent}
\newcommand{\dsp}{\displaystyle}
\newcommand{\ind}{{{\large 1} \hspace*{-1.6mm} {\large 1}}}


\textheight 25cm
\textwidth 16cm
\oddsidemargin 0cm
\evensidemargin 0cm
\topmargin 0cm
\hoffset -0mm
\voffset -20mm

\pagestyle{plain}

\begin{document}

\section*{Question 8}

Le laplacien s'exprime en coordonn�es polaires de la mani�re suivante :

$$
\Delta w = \frac{1}{r} \frac{\partial}{\partial r} (r \frac{\partial w}{\partial r}) +
\frac{1}{r^2} \frac{\partial^2 w}{\partial \theta^2} =
\frac{\partial^2 w}{\partial r^2} + \frac{1}{r} \frac{\partial w}{\partial r} + \frac{1}{r^2} \frac{\partial^2 w}{\partial \theta^2}
$$

En effet :

$$
\begin{aligned}
  \frac{\partial^2 w}{\partial x^2} + \frac{\partial^2 w}{\partial y^2} &=
  \frac{\partial}{\partial x} (\frac{\partial w}{\partial r} \frac{\partial r}{\partial x} + \frac{\partial w}{\partial \theta} \frac{\partial \theta}{\partial x}) + \frac{\partial}{\partial y} (\frac{\partial w}{\partial r} \frac{\partial r}{\partial y} + \frac{\partial w}{\partial \theta} \frac{\partial \theta}{\partial y}) \\
  &= \frac{\partial}{\partial x} (\frac{\partial w}{\partial r} \frac{x}{\sqrt{x^2 + y^2}} + \frac{\partial w}{\partial \theta} \frac{-y}{x^2 + y^2}) + \frac{\partial}{\partial y} (\frac{\partial w}{\partial r} \frac{y}{\sqrt{x^2 + y^2}} + \frac{\partial w}{\partial \theta} \frac{x}{x^2 + y^2})
\end{aligned}
$$

Or :

$$
\left\{\begin{aligned}
    \frac{\partial w}{\partial r} &= \frac{1}{a} \frac{\partial w}{\partial \eta} \\
    \frac{\partial^2 w}{\partial r^2} &= \frac{1}{a^2} \frac{\partial^2 w}{\partial \eta^2} \\
    \frac{\partial^2 w}{\partial t^2} &= \frac{c^2}{a^2} \frac{\partial^2 w}{\partial \tau^2} \\
                                      &= \frac{T}{\rho a^2} \frac{\partial^2 w}{\partial \tau^2}
\end{aligned}\right.
$$

Donc :

$$
\begin{aligned}
  \rho \dfrac{\partial^2 w}{\partial t^2} = T \Delta w
  &\Leftrightarrow
  \rho \frac{T}{\rho a^2} \frac{\partial^2 w}{\partial \tau^2} = T(\frac{1}{a^2} \frac{\partial^2 w}{\partial \eta^2} + \frac{1}{a\eta} \frac{1}{a} \frac{\partial w}{\partial \eta} + \frac{1}{\eta^2 a^2} \frac{\partial^2 w}{\partial \theta^2}) \\
  &\Leftrightarrow
  \frac{\partial^2 w}{\partial \tau^2} = \frac{\partial^2 w}{\partial \eta^2} + \frac{1}{\eta} \frac{\partial w}{\partial \eta} + \frac{1}{\eta^2} \frac{\partial^2 w}{\partial \theta^2}
\end{aligned}
$$

\section*{Question 9}

$w$ est deux fois d�rivable par rapport � chacun de ses param�tres. On peut donc
lui appliquer un d�veloppement de Taylor :

$$
\left\{\begin{aligned}
  w(\eta_{i+1}, \theta_j, \tau_n) &= w(\eta_i + d\eta, \theta_j, \tau_n) \\
                                  &= w(\eta_i, \theta_j, \tau_n) + d\eta \frac{\partial w}{\partial \eta}(\eta_i, \theta_j, \tau_n) + \frac{d\eta^2}{2} \frac{\partial^2 w}{\partial \eta^2}(\eta_i, \theta_j, \tau_n) + \mathcal{O}(h^2) \\
  w(\eta_{i-1}, \theta_j, \tau_n) &= w(\eta_i - d\eta, \theta_j, \tau_n) \\
                                  &= w(\eta_i, \theta_j, \tau_n) - d\eta \frac{\partial w}{\partial \eta}(\eta_i, \theta_j, \tau_n) + \frac{d\eta^2}{2} \frac{\partial^2 w}{\partial \eta^2}(\eta_i, \theta_j, \tau_n) + \mathcal{O}(h^2)
\end{aligned}\right.
$$

Donc :

$$
\left\{\begin{aligned}
    \frac{w(\eta_{i+1}, \theta_j, \tau_n) - w(\eta_{i-1}, \theta_j, \tau_n)}{2d\eta} &= \frac{\partial w}{\partial \eta}(\eta_i, \theta_j, \tau_n) + \mathcal{O}(h^2) \\
    \frac{w(\eta_{i+1}, \theta_j, \tau_n) - w(\eta_i, \theta_j, \tau_n) + w(\eta_{i-1}, \theta_j, \tau_n)}{d\eta^2} &= \frac{\partial^2 w}{\partial \eta^2}(\eta_i, \theta_j, \tau_n) + \mathcal{O}(h^2)
\end{aligned}\right.
$$

On a un r�sultat analogue pour les d�riv�es par rapport � $\theta$ et $\tau$, donc on en d�duit l'�quation discr�tis�e.

\section*{Question 10}

\begin{tabular}{|c|c|}
  \hline
  Equation & Discr�tisation \\
  \hline
  $w(1, \theta, \tau) = 0$ & $w_{1, j}^n = 0$ \\
  \hline
  $w(\eta,\theta,0) = w_0(\eta \cos(\theta),\eta \sin (\theta))$ & $w_{i, j}^0 = w_0(\eta_i \cos(\theta_j),\eta_i \sin (\theta_j))$ \\
  \hline
  $\dfrac{\partial w(\eta, \theta, 0)}{\partial \tau} = 0$ & $\dfrac{w_{i, j}^1 - w_{i, j}^{-1}}{2d\tau} = 0$ \\
  \hline
\end{tabular}

\section*{Question 11}



\section*{Question 12}



\section*{Question 13}



\section*{Question 14}



\section*{Question 15}



\section*{Question 16}



\section*{Question 17}



\section*{Question 18}



\section*{Question 19}



\section*{Question 20}



\end{document}
